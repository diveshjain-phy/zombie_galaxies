
\documentclass[12pt]{article}

\usepackage[utf8]{inputenc}
\usepackage{amsmath,mathtools}
\usepackage{tcolorbox}
\newcommand{\mbf}[1]{\mathbf{#1}}
\newcommand{\tbf}[1]{\textbf{#1}}
\newcommand{\dsum}[3]{$\sum^{#1}_{#2}{#3}$}
\newcommand{\dint}[3]{\int^{#1}_{#2}{#3}}
\newcommand{\tit}[1]{\textit{#1}}
\newcommand{\fn}[1]{\footnote{#1}}
\newcommand{\de}[2]{\frac{d{#1}}{d{#2}}}
\newcommand{\ch}[2]{\Gamma^{#1}_{#2}}
\newcommand{\chris}{\ch{\mu}{\alpha \beta}=\frac{1}{2}g^{\mu \lambda}(\p_{\alpha} g_{\beta \lambda}+\p_\beta g_{\alpha \lambda} - \p_\lambda g_{\alpha \beta})}
\newcommand{\p}{\partial}
\newcommand{\pe}[2]{\frac{\partial{#1}}{\partial{#2}}}
\newcommand{\n}{\nonumber}
\newcommand{\cbox}{tcolorbox}
\newcommand{\cc}[1]{\left({#1}\right)}
\newcommand{\rr}[1]{\left[{#1}\right]}
\newcommand{\vd}[1]{\dot{\vec{#1}}}
\newcommand{\tx}[1]{\text{#1}}
\begin{document}
\title{List Of Phoenix Sources}
\author{Divesh Jain}
\maketitle
\newpage
\section{List Of Sources}
Below I present the list of sources by unifying data from DB1-$www.galaxyclusters.com$ and DB2(Bold font)-$https://arxiv.org/pdf/1808.04057.pdf
$
\begin{itemize}
\item The list is organized in ascending order of Redshift .\\
\item First column indicates designation of source.Second column represents Redshift as provided in DB1  and Third column represents Redshift from DB2.
\end{itemize}

\begin{center}
\begin{tabular}{|p{4cm}|p{4cm}|p{4cm}|}
\hline 
\tbf{Designation} & Redshift1(z) & Redshift2\\
\hline
ACO S 753 & 0.0130 & \tbf{0.014}\\
A4038 & 0.0303 & \tbf{0.02819}\\
\tbf{A2063}&&\tbf{0.0349}\\
\tbf{A548b-NW} &&\tbf{0.0424}\\
\tbf{A548b-N} &&\tbf{0.0424}\\
A85& 0.0557 & \tbf{0.0551}\\
A133 & 0.0603\\
A725 & 0.0900\\
A13 & 0.0943 & \tbf{0.0943}\\
A2048 &0.0980& \tbf{0.0972}\\
A2443 & 0.1080 & \tbf{0.1080}\\
A1033 & 0.1220\\
\tbf{A1664} & &\tbf{0.1283}\\
24P73 & 0.1500\\
\hline
\end{tabular}
\end{center}
\tbf{Comments:}
\begin{itemize}
\item \tit{The galaxy cluster database even if has classified the above as phoenix sources, There is a special mention of 'candidates' for few of the above sources in the surface brightness column, of which there is no description.}\\
\item \tit{ The Redshift is measured from SDSS data. How do we have a discrepancy in redshift for some of the phoenix sources? How much important is redshift for us?}
\end{itemize}

\end{document}